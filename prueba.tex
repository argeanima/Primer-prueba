\documentclass{article}

\usepackage[utf8]{inputenc}
\usepackage[spanish]{babel}
\usepackage{graphicx}
\usepackage{amsmath}
\title{Modelos matemáticos discretos}
\author{Uriel Alejandro Nolasco Hernández}	
\begin{document}
	\maketitle
	\section{Ecuaciones en diferencias}

	\subsection{Primer Orden}
	Tenemos \$1000 que vamos a invertir a un interés del 1\% mensual.
El valor de la inversión cuando han tanscurrido $n$ meses es $$x_n=1000(1.01)^n$$
$$\lim_{x\to\infty}\frac{1}{x}=0$$
	
	\subsection{Segundo Orden}
\begin{center}
\includegraphics[width=8cm]{grafica}	
\end{center}
\begin{center}
	\begin{tabular}{|c|r|}
\hline
mes & valor\\
\hline
\hline
0 & 1000\\
1 & 1010\\
\hline
	\end{tabular}
	
\end{center}
Calcular los valores propios de $A=(
\begin{matrix}
1 & 2\\
\pi & 4\\ 
\end {matrix}
)
$
\end{document}

